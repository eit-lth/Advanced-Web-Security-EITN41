\documentclass{article}
\usepackage{graphicx}
\usepackage{fullpage}
\usepackage{booktabs}
\usepackage{listings}
\usepackage{url}
\usepackage[hidelinks]{hyperref}
\usepackage{amssymb}


\begin{document}
	
\title{Home Assignment 3}
\author{Advanced Web Security}
\date{\the\year}

\maketitle

\section*{B-assignments}
\textbf{For grade 3, complete the three B-assignments below and solve them in groups of $\leq 2$ students.}

\begin{description}
	\item[B-1]{Assume that we have a commitment scheme $x = h(v, k)$, where $v$ is a 1-bit commitment,
		$k$ is a $K$-bit random string and $h$ is a hash function with output truncated to $X$ bits.
		Fix $K$ to be 16 bits. For different appropriate choices of $X$, simulate the following.
		\begin{itemize}
			\item[1)] The probability of breaking the binding property of the scheme. Note that we assume a non-idiot attacker. This means that the attacker optimizes her chances of success by making sure to choose a value for $k$ that she knows can break the binding property \textit{before} she commits to $v$. This attack is closely related to a well-known property for hash functions.
			\item[2)] The probability of breaking the concealing property of the scheme. Here the attack is carried out by the receiver of the commitment and it is thus not possible to choose optimum values before the commitment is made. This attack is also closely related to a well-known property for hash functions.
		\end{itemize}
		Write a short report that summarizes your work and your findings. Make sure that your report includes at least the following aspects.
		\begin{itemize}
			\item Describe in general terms how an attacker would proceed to perform the two attacks.
			\item Clearly present your algorithms for computing the probabilities.
			\item What can you say about how the probabilities varies with $X$? Plot graphs illustrating the probabilities. 
			\item Identify the hash function properties that the two attacks relate to. What is the relation? Treat both cases separately. A hint is that the birthday paradox is closely related to one of the two cases.
		\end{itemize}
		To get on the right track, it can be useful to consider two column vectors, representing all possible values for $x$:	
		\begin{center}
		\begin{tabular}{c|c}
			$v=0$ & $v=1$ \\\hline
			$h(0,0)$ & $h(1,0)$ \\
			$h(0,1)$ & $h(1,1)$ \\
			$h(0,2)$ & $h(1,2)$ \\
			\vdots & \vdots \\
			$h(0,2^{16}-1)$ & $h(1,2^{16}-1)$ \\
		\end{tabular}
		\end{center}
		Now, what is required for breaking the binding property? What is required for breaking the concealing property?\\\\
		
		\textbf{Assessment}:
		\begin{itemize}
			\item Upload your report to Urkund, jonathan.sonnerup.lu@analys.urkund.se. 
			Only one student per group must do this.
			\item Upload your code to Urkund, jonathan.sonnerup.lu@analys.urkund.se. Only one student per group must do this.
			\item Upload your report to Moodle (it will be manually graded). Both students must do this.
		\end{itemize}}
		
		
		\item[B-2]{A recently elected, semi-competent, leader of a large country accidentally initiated a nuclear missile launch. Your task is to prevent disaster by deactivating the detonation circuit. The deactivation process
			utilizes a (k,n) threshold scheme, and you need to provide the master secret of the scheme.
			
			Implement a program that takes as input
			\begin{itemize}
				\item[-] parameters $k$ and $n$ with $3\leq k<n\leq 8$,
				\item[-] your private polynomial,
				\item[-] polynomial shares from collaborating participants.
			\end{itemize}
			The program output should be the deactivation code (an integer).
			
			\textbf{Example:}\\
			You are participant 1 out of 8 in a (5,8) threshold scheme.
			All participants have each chosen a private polynomial of degree 4.
			The secret master polynomial is simply the sum of all your individual private polynomials, so that
			\[f(x) = f_{1}(x) + f_{2}(x) + ... + f_{n}(x),\]
			and the master secret is the constant term (an integer) of this polynomial.
			
			Your private polynomial is $f_{1}(x) = 13 +  8x + 11x^2 +  1x^3 +  5x^4$.\\
			You have generously shared points on your polynomial, one with each other participant;
			\begin{eqnarray*}
				f_{1}(2) &=& 161,\\
				f_{1}(3) &=& 568,\\
				f_{1}(4) &=& 1565,\\
				f_{1}(5) &=& 3578,\\
				f_{1}(6) &=& 7153,\\
				f_{1}(7) &=& 12956,\\
				f_{1}(8) &=& 21773.
			\end{eqnarray*}
			You have also been given shares from the other participants' polynomials, one from each participant;
			\begin{eqnarray*}
				f_{2}(1) &=& 75,\\
				f_{3}(1) &=& 75,\\
				f_{4}(1) &=& 54,\\
				f_{5}(1) &=& 52,\\
				f_{6}(1) &=& 77,\\
				f_{7}(1) &=& 54,\\
				f_{8}(1) &=& 43.
			\end{eqnarray*}
			
			Collaborating with participants 2, 4, 5 and 7, they reveal their points on the master polynomial to you;
			\begin{eqnarray*}
				f(2) &=& f_{1}(2) + ... + f_{8}(2) = 2782,\\
				f(4) &=& f_{1}(4) + ... + f_{8}(4) = 30822,\\
				f(5) &=& f_{1}(5) + ... + f_{8}(5) = 70960,\\
				f(7) &=& f_{1}(7) + ... + f_{8}(7) = 256422.
			\end{eqnarray*}
			
			The deactivation code in this case is 110.
			
			\textbf{Assessment}:
			\begin{itemize}
				\item Upload your code to Urkund, jonathan.sonnerup.lu@analys.urkund.se. Only one student per group must do this.
				\item There will be a Moodle question following the problem statement above. Both students must finish the Moodle quiz. There will be	a test quiz on Moodle where you can try your implementation as many times as you like. The test quiz will not be graded.
			\end{itemize}}
			
			\item[B-3]{Several topics and problems encountered in the course so far are related to the notions of \textit{semantic security} and \textit{malleability} of cryptosystems. This is also related to the ciphertext indistinguishability properties of cryptosystems, i.e., \textit{IND-CPA}, \textit{IND-CCA} and \textit{IND-CCA2}. Read about these different notions and understand their properties, definitions and how they apply to RSA and ElGamal encryption. You are free to use any source you wish, but the Wikipedia entries are very good and sufficient for our purpose.\\
				\textbf{Assessment}:
				\begin{itemize}
					\item There will be one Moodle-question with statements about differences. Four of these statements are correct and your task is to identify these.  
					
					You will receive 1 point for each correct answer and -1 point for each error. You will need 3 points to pass. Thus, your best bet is to either pick 3 choices that you are absolutely certain are correct, or pick 5 choices among which you know that you have picked the 4 correct ones.
					
					It can be a good idea to have the material accessible when you answer the questions, but it is extremely recommended that you read and understand them it advance. Both students must finish the quiz.
				\end{itemize}}
				
				
			\end{description}
		
			\clearpage

			\section*{C-Assignments}
			\textbf{For grade 4, complete the C-assignment below and solve it in groups of $\leq 2$ students.}
			
			\begin{description}
				\item[C-1]{In a presidential election, the people can vote on one of two candidates. Let us call them Grump and Flinton. The voting system is a new e-voting system deployed by your company. Being the
					security engineer, it is your job to implement a function that counts the votes. The votes are encrypted
					using the Paillier cryptosystem\footnote{\url{https://en.wikipedia.org/wiki/Paillier_cryptosystem}}. Since the votes are anonymous, you are not allowed to decrypt a single vote.
					Rather, you have to utilize the homomorphic property of Paillier:
					\[
					E(v_1) \cdot E(v_2) = E(v_1 + v_2)
					\]	
					to get the total sum of votes. A vote for Mr. Grump is encoded as a +1, whereas a vote for Mrs. Flinton
					is encoded as -1. If the sum of all votes is positive, Mr. Grump wins and vice versa. Note that in
					$\mathbb{Z}_n$, the number ``$-x$'' is written as ``$n-x$''.
					
					Implement a program that takes as input
					\begin{itemize}
						\item[-] Two prime numbers $p$ and $q$.
						\item[-] An element $g \in \mathbb{Z}_{n^2}^{*}$.
						\item[-] A file containing the encrypted votes, one per line.
					\end{itemize}
					
					The program output should be the sum of the votes, e.g. 5, -3.
					
					\textbf{Example:}\\
					We have three voters and all voted for Mrs. Flinton. The primes used are
					$(p, q) = (5, 7)$, $g = 867$, and we have the following (integer) ciphertexts:
					\begin{verbatim}
					929
					296
					428
					\end{verbatim}
					The product of the ciphertexts is $c = c_1 \cdot c_2 \cdot c_3 = 52 ~(\textrm{mod } n^2)$
					which gives the sum of votes, $v_{tot} = 32 = -3 ~(\textrm{mod } n)$
					
					\textbf{Note:} Plaintexts are reduced mod $n$, ciphertexts are reduced mod $n^2$.
					
					\textbf{Assessment}:
					\begin{itemize}
						\item Upload your code to Urkund, jonathan.sonnerup.lu@analys.urkund.se. Only one student per group must do this.
						\item There will be a Moodle question following the problem statement above. Both students must finish the Moodle quiz. There will be	a test quiz on Moodle where you can try your implementation as many times as you like. The test quiz will not be graded.
					\end{itemize}}
					
				\end{description}
				
\end{document}
