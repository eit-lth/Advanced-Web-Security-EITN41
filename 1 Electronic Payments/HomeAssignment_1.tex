\documentclass{article}
\usepackage{graphicx}
\usepackage{fullpage}
\usepackage{booktabs}
\usepackage{listings}
\usepackage{url}
\usepackage[hidelinks]{hyperref}
\usepackage{amssymb}


\begin{document}
	
	\title{Home Assignment 1}
	\author{Advanced Web Security}
	\date{\the\year}
	
	\maketitle
	
	\section*{B-assignments}
	\textbf{For grade 3, complete the three B-assignments below and solve them in groups of $\leq 2$ students.}
	
	\begin{description}
		
		\item[B-1]{Implement the Luhn algorithm in your favorite language. You will be given
			a list of card numbers with one digit censored with an ``X''. Use your implementation to find the
			censored digit in order for the card number to be valid. You will be given a text file,  consisting of 100 card numbers,
			one per line. The answer is the concatenation of all censored digits, 
			in order according to the list.
			
			\textbf{Example:} Let the input list be
			\begin{verbatim}
			12774212857X4109
			586604X108627571
			7473X86953606632
			4026467X45830632
			20X3092648604969
			\end{verbatim}
			The answer is given by the numeric string ``54963''.\\
			\textbf{Assessment}:
			\begin{itemize}
				\item Upload your code to Urkund, jonathan.sonnerup.lu@analys.urkund.se.
				\item There will be one Moodle questions following the problem statement above. Both students must finish the Moodle quiz.	There will be a test quiz on Moodle, where you can try your implementation as many times as you like. The test quiz will not be graded.
			\end{itemize}
		}
		
		\item[B-2]{Read sections 4 and 4.1 in the MicroMint paper\footnote{R. L. Rivest and A. Shamir - PayWord and MicroMint: Two simple micropayment schemes}.
			Write a program that simulates the time needed to generate MicroMint coins.
			Model the process as a balls-and-bins problem.
			Your program should take three parameters; $u$, $k$ and $c$.
			The parameter $u$ is the number of bits used for identifying the bin, so there are $b=2^u$ bins.
			Parameter $k$ is the number of collisions (balls in the same bin) needed to make a coin, so that fewer than $k$ balls in a bin make no coins, 
			and $k$ or more balls in a bin make precisely one coin.
			All bins are empty at first.
			At each iteration, throw a ball into a randomly selected bin.
			How many iterations (how many balls thrown) does it take before you generate $c = 1, 100$ and 10,000 coins, respectively?
			You will be given $u, k \textrm{ and } c$.\\
			\textbf{Note}: With respect to the MicroMint paper, use $t=0$.\\
			The above procedure describes one simulation.
			You will need to run several such simulations in order to compute \textbf{the averages} reliably.
			Construct 99.9\% confidence intervals for the three mean values.
			So, you may ask, for how long do I need to run these simulations?
			You will be given appropriate widths for your confidence intervals.
			With $n$ statistical observations $x_1, x_2,\ldots, x_n$, a 99.9\% confidence interval for the mean is given by
			\[\left(\bar{x}\pm\lambda\cdot\frac{s}{\sqrt{n}}\right),\]
			where $\bar{x}$ and $s$ are your sample-estimated average and standard deviation, respectively.
			You may use $\lambda=3.66$.
			Below we present the minimum and maximum values for $u$ and $k$
			that you should be able to handle. Make sure you can handle all integer values between these extremes. \textbf{Note}: A simulation should take no longer than 30 minutes.
			
			\textbf{Example}: In the tables below you are given two examples of simulations results. 
			In the first example,
			the simulation is run with the parameters $u = 16, k = 2, c = 1$. The mean is calculated to be 322.
			Note that this value is just an estimate of the \textit{true} mean value. 
			If you create a confidence interval of width 22 (in this case) for the mean, 
			then you will reliably assess the true mean value, which will be enough to pass this assignment.
			
			\begin{table}[h]
				\centering
				\caption{Simulation with $u = 16, k = 2$.}
				\begin{tabular}{lrr}
					\toprule
					\# coins ($c$) & mean ($\bar{x}$) & c.i. width \\
					\midrule
					1      &        322 &       22 \\
					
					100    &       3685 &       24 \\
					
					10k    &      45270 &       22 \\
					\bottomrule
				\end{tabular}
			\end{table}
			
			\begin{table}[h]
				\centering
				\caption{Simulation with $u = 20, k = 7$.}
				\begin{tabular}{lrr}
					\toprule
					\# coins ($c$) & mean ($\bar{x}$) & c.i. width \\
					\midrule
					1      &     493981 &    79671 \\
					100    &    1069997 &    15616 \\
					10k    &    2420113 &     4783 \\
					\bottomrule
				\end{tabular}
			\end{table}
			
			\textbf{Assessment}:
			\begin{itemize}
				\item Upload your code to Urkund, jonathan.sonnerup.lu@analys.urkund.se.
				\item There will be one Moodle question following the problem statement above. Both students must finish the Moodle quiz.	There will be a test quiz on Moodle, where you can try your implementation as many times as you like. The test quiz will not be graded.
			\end{itemize}
		}
		
		\item[B-3] Lightweight Bitcoin clients, so-called Simplified Payment Verification (SPV) nodes, that only store the block headers may wish to verify that a transaction is included in the blockchain. To do this, the SPV node queries a full node for the Merkle path of a certain transaction. The path is used to compute the Merkle root, which then can be compared to the Merkle root stored in block header. In this problem, you will take on the role of both an SPV node that verifies the transaction using the Merkle path, and also a full node providing the Merkle path for a given transaction.
		
		{\bf Part 1, SPV node.}\\ Implement a program that takes a file with a leaf node $L$, given as a hexadecimal string, and a Merkle path as input, with each node in the path on a separate line. The $2t$-byte strings should be interpreted as $t$-byte byte arrays. The Merkle path nodes are given in the order of highest depth first, i.e., the leaf node sibling. By convention, the root node is at depth $0$ and the leaves are at depth $\lceil \log_2 n\rceil$, where $n$ is the number of leaves. Each string representing nodes in the Merkle path is preceded by the letter 'L' or 'R', indicating if the sibling node in the path is a \textit{Left} or \textit{Right} node. The program should output the Merkle root as a hexadecimal string. Use SHA-1 as hash function throughout the tree and merge nodes using concatenation of SHA-1 results (byte arrays).
		
		{\bf Example} Using a file of the format
		\begin{center}
			\begin{tabular}{l}
				\texttt{2354cf006ef4eeefeddf29b9e68d5cb1918ed589}\\
				\texttt{R69968f8d734080390646bd0f3afff78baadebd2b}\\
				\texttt{Led64e17870e63f55b71542f0818ff7639b1f9985}\\
				\texttt{L7a6ba60c80a893b7a02999b6415c6ec67d5883b4}\\
				\texttt{L64b64c7760e5559aefe701790ee0564af6458cb4}\\
				\texttt{L53f1eab7ccd09600908bc49044669cd8fc996171}\\
				\texttt{Rc5684eb22d8745a777037c19ff3eff85be800334}\\
				\texttt{L058e2c0d7d103a7b45b2a4408ac3389eb10048fe}
			\end{tabular}
		\end{center}
		the node \texttt{2354cf006ef4eeefeddf29b9e68d5cb1918ed589} is concatenated with the first node in the path, \texttt{69968f8d734080390646bd0f3afff78baadebd2b}. The SHA-1 hash if this concatenation is then appended to \texttt{ed64e17870e63f55b71542f0818ff7639b1f9985} and so on. The resulting Merkle root is \texttt{6f51120bc17e224de27d3d27b32f05d0a5ffb376}.
		
		
		{\bf Part 2, Full node.}\\ Implement a program that takes an integer index $i$, another integer index $j$, and a set of leaves, ($\ell(0),\ell(1),\ell(2),\ldots,\ell(n-1)$), in a Merkle tree. The leaves are given as hexadecimal strings as input, but should be interpreted as byte arrays. The input is summarized in a file, starting with the integer index $i$, the integer index $j$, and then followed by the leaves, one input on each line. Your program should provide: 
		\begin{itemize}
			\item[1.] The Merkle path for leaf $\ell(i)$, starting with the sibling of $\ell(i)$.
			\item[2.] The Merkle path node at a given depth $j$ (this will be used in the assessment). 
			\item[3.] The resulting Merkle root (this will be used in the assessment).
		\end{itemize}
		Specifically, the output of your program should provide a concatenation of items 2 and 3 above. Make sure you support an arbitrary number of leaves. The format of the nodes should be hexadecimal strings. Each string representing nodes in the Merkle path must be preceded by the letter 'L' or 'R', indicating if the sibling node in the path is a \textit{Left} or \textit{Right} node. While building the tree, starting with the leaves, if the set of nodes on depth $k$ is odd, simply append a copy of the last node to get an even number of nodes.
		
		{\bf Example} Using a file of the format
		\begin{center}
			\begin{tabular}{l}
				\texttt{9}\\
				\texttt{2}\\
				\texttt{fcfdd2a12d8b0b75c2edb47d691470dd0178c566}\\
				\texttt{031e3057d40d25472cdb80e4952ed6738936aba8}\\
				\texttt{7164148d695971de1b9f31be8e730299823b1def}\\
				\texttt{64d866d14053a33e75744d966daa4f47b92018b9}\\
				\texttt{a7e62ee4822cb29ec7004841d8f819965be169d9}\\
				\texttt{c176f7a3dc7a3f59136b42163efa8f6350bb073c}\\
				\texttt{c25053cedc71463287168687accf4e701a7ace0f}\\
				\texttt{2af15e40605eb51d3a2e041f9799f3a070398eb1}\\
				\texttt{053dfa25bf07b0d530c6bb9f803d60062f4e0e00}\\
				\texttt{6c136b269efa9535f02732d68c80371625783cfc}\\
				\texttt{d1705b195fb31e4e9c0b03570c677e59a19250b8}\\
				\texttt{e5fb3dc4357332ec108e5d4d25991942e4ec9a95}
			\end{tabular}
		\end{center}
		will result in a Merkle tree with the leaves at depth 4. The Merkle path for the node with index 9, i.e., \texttt{6c136b269efa9535f02732d68c80371625783cfc}, is given by 
		\begin{center}
			\begin{tabular}{l}
				\texttt{L053dfa25bf07b0d530c6bb9f803d60062f4e0e00}\\
				\texttt{R9befe43f5e8e4dce20616129c9fa1b782c080e73}\\
				\texttt{R8d3f164890509c6510cc9bc975cb978f0b872fbb}\\
				\texttt{Laa34d87b649005b70e9ea55390679e84c7627d1c}\\
			\end{tabular}
		\end{center}
		These nodes are on depth 4, 3, 2 and 1 respectively (starting at the top of the list). Thus, the Merkle path node at depth $2$ is \texttt{8d3f164890509c6510cc9bc975cb978f0b872fbb} and this node is Right sibling. The Merkle root is given by \texttt{1781a6ea9a22f67e8a09cb54bbdc6d99d0efc081}. The concatenation of these is given by
		\begin{center}
			\texttt{R8d3f164890509c6510cc9bc975cb978f0b872fbb1781a6ea9a22f67e8a09cb54bbdc6d99d0efc081}
		\end{center}
		Note that the R has been added to the answer.
		
		\textbf{Assessment}:
		\begin{itemize}
			\item Upload your code to Urkund, jonathan.sonnerup.lu@analys.urkund.se.
			\item There will be three Moodle question following the problem statement above (one Merkle path node and two Merkle roots). Both students must finish the Moodle quiz.	There will be a test quiz on Moodle, where you can try your implementation as many times as you like. The test quiz will not be graded.
		\end{itemize}
		
		
	\end{description}
	
	\clearpage
	
	\section*{C-Assignment}
	\textbf{For grade 4, complete the C-assignment below and solve it in groups of $\leq 2$ students.}
	
	\begin{description}
		\item[C-1]{Implement the coin withdrawal (the version with 2k quadruples) in the untraceable e-cash
			scheme given in the lecture notes.
			\begin{itemize}
				\item You may use a variant of RSA with easily manageable numbers. Note though that too small numbers might get you in trouble when you try to unblind if you are not careful.
				\item The data transfer can be simulated locally inside the program.
				\item The extended Euclidean algorithm can be used to find the inverse of 3 mod $x$. You are free to use existing code for this algorithm as long as you reference it properly.
				\item Choose sensible functions $f$ and $h$.
				\item Be careful when choosing your $r_i$. Not all choices are suitable.
			\end{itemize}
			
			\textbf{Assessment}:
			\begin{itemize}
				\item Summarize your work in a short report, making it clear that the program works as intended.
				\item Upload your code to Urkund, jonathan.sonnerup.lu@analys.urkund.se. Only submit it once.
				\item Upload your report to Urkund, jonathan.sonnerup.lu@analys.urkund.se. Only submit it once.
				\item Upload your report to Moodle (it will be manually graded). BOTH students must do this.
			\end{itemize}
		}
	\end{description}
\end{document}
